%%%%%%%%%%%%%%%%%%%%%%%%%%%%%%%%%%%%%%%%%%%%%%%%%%%%%%%%%%%%%%%%%%%%%%
%
% Institut für Rechnergestuetzte Automation
% Forschungsgruppe Industrial Software
% Arbeitsgruppe ESSE
% http://security.inso.tuwien.ac.at/
% lva.security@inso.tuwien.ac.at
%
% Version 2012-10-17
% 
%%%%%%%%%%%%%%%%%%%%%%%%%%%%%%%%%%%%%%%%%%%%%%%%%%%%%%%%%%%%%%%%%%%%%%

\documentclass[12pt,a4paper,titlepage,oneside]{scrartcl}
\usepackage{esseProtocol}

%%%%%%%%%%%%%%%%%%%%%%%%%%%%%%%%%%%%%%%%%%%%%%%%%%%%%%%%%%%%%%%%%%%%%%
%
% FOR STUDENTS
%
%%%%%%%%%%%%%%%%%%%%%%%%%%%%%%%%%%%%%%%%%%%%%%%%%%%%%%%%%%%%%%%%%%%%%%

% Group number or "0" for Lab0
\newcommand{\gruppe}{5}
% Date
\newcommand{\datum}{05.06.2013}
% valid values: "Lab0", "Lab1" (be sure to use Uppercase for first character)
\newcommand{\lab}{Lab1}

% name of course, for example: "IT Security in Large IT Infrastructures", "Security for Systems Engineering", "Introduction to Security"
\newcommand{\lvaname}{IT Security in Large IT Infrastructures}
% number of course, for example: "183.633", "183.637", "183.594"
\newcommand{\lvanr}{183.633}
% year and term, for example: "SS 2012", "WS 2012", "SS 2013", etc.
\newcommand{\semester}{SS 2013}

% Student data in Lab0 or 1. student of group in Lab1
\newcommand{\studentAName}{Michael Heil}
\renewcommand{\studentAMatrnr}{0826358}
\newcommand{\studentAEmail}{e0826358@student.tuwien.ac.at}

% 2. student of group in Lab1, for Lab0 or if your group has less students, remove this 3 lines
\newcommand{\studentBName}{Lukas Puschmann}
\renewcommand{\studentBMatrnr}{0825354}
\newcommand{\studentBEmail}{e0825354@student.tuwien.ac.at}

% 3. student of group in Lab1, for Lab0 or if your group has less students, remove this 3 lines
\newcommand{\studentCName}{Dominik Amon}
\renewcommand{\studentCMatrnr}{1228536}
\newcommand{\studentCEmail}{e1228536@student.tuwien.ac.at}


%%%%%%%%%%%%%%%%%%%%%%%%%%%%%%%%%%%%%%%%%%%%%%%%%%%%%%%%%%%%%%%%%%%%%%
%
% DO NOT CHANGE THE FOLLOWING PART
%
%%%%%%%%%%%%%%%%%%%%%%%%%%%%%%%%%%%%%%%%%%%%%%%%%%%%%%%%%%%%%%%%%%%%%%

\newcommand{\dokumenttyp}{Report \lab}

\begin{document}

\maketitle
\setcounter{section}{0}
\setcounter{tocdepth}{2}
\tableofcontents

%%%%%%%%%%%%%%%%%%%%%%%%%%%%%%%%%%%%%%%%%%%%%%%%%%%%%%%%%%%%%%%%%%%%%%
%
% CONTENT OF DOCUMENT STARTS HERE
%
%%%%%%%%%%%%%%%%%%%%%%%%%%%%%%%%%%%%%%%%%%%%%%%%%%%%%%%%%%%%%%%%%%%%%%

\section{lab1a - Safety Objectives}
Security is the core requirement to this software system. In this chapter we will list and explain the
safety objectives to the system as a whole as well as its implications to its separate components. The order
of the targets has no implications to their importance.

\subsection{Authorization}


\subsection{Authenticity}


\subsection{Data Integrity}


\subsection{Availability}


\subsection{Confidentiality}


\subsection{Non-Repudiation}


\section{lab1b - Interface Definitions}

\section{lab1c - Security Analysis}


%%%%%%%%%%%%%%%%%%%%%%%%%%%%%%%%%%%%%%%%%%%%%%%%%%%%%%%%%%%%%%%%%%%%%%
%
% DO NOT CHANGE THE FOLLOWING PART
%
%%%%%%%%%%%%%%%%%%%%%%%%%%%%%%%%%%%%%%%%%%%%%%%%%%%%%%%%%%%%%%%%%%%%%%

\end{document}


